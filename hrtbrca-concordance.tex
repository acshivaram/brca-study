\Sconcordance{concordance:hrtbrca.tex:hrtbrca.Rnw:%
1 7 1 49 0 10 1 4 0 18 1 18 0 18 1 20 0 6 1 6 0 17 1 18 0 19 1 9 0 %
12 1 6 0 14 1 4 0 7 1 6 0 10 1 6 0 2 1}
