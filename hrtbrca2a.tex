\documentclass[]{article}
\usepackage{lmodern}
\usepackage{amssymb,amsmath}
\usepackage{ifxetex,ifluatex}
\usepackage{fixltx2e} % provides \textsubscript
\ifnum 0\ifxetex 1\fi\ifluatex 1\fi=0 % if pdftex
  \usepackage[T1]{fontenc}
  \usepackage[utf8]{inputenc}
\else % if luatex or xelatex
  \ifxetex
    \usepackage{mathspec}
    \usepackage{xltxtra,xunicode}
  \else
    \usepackage{fontspec}
  \fi
  \defaultfontfeatures{Mapping=tex-text,Scale=MatchLowercase}
  \newcommand{\euro}{€}
\fi
% use upquote if available, for straight quotes in verbatim environments
\IfFileExists{upquote.sty}{\usepackage{upquote}}{}
% use microtype if available
\IfFileExists{microtype.sty}{%
\usepackage{microtype}
\UseMicrotypeSet[protrusion]{basicmath} % disable protrusion for tt fonts
}{}
\usepackage[margin=1in]{geometry}
\usepackage{color}
\usepackage{fancyvrb}
\newcommand{\VerbBar}{|}
\newcommand{\VERB}{\Verb[commandchars=\\\{\}]}
\DefineVerbatimEnvironment{Highlighting}{Verbatim}{commandchars=\\\{\}}
% Add ',fontsize=\small' for more characters per line
\usepackage{framed}
\definecolor{shadecolor}{RGB}{248,248,248}
\newenvironment{Shaded}{\begin{snugshade}}{\end{snugshade}}
\newcommand{\KeywordTok}[1]{\textcolor[rgb]{0.13,0.29,0.53}{\textbf{{#1}}}}
\newcommand{\DataTypeTok}[1]{\textcolor[rgb]{0.13,0.29,0.53}{{#1}}}
\newcommand{\DecValTok}[1]{\textcolor[rgb]{0.00,0.00,0.81}{{#1}}}
\newcommand{\BaseNTok}[1]{\textcolor[rgb]{0.00,0.00,0.81}{{#1}}}
\newcommand{\FloatTok}[1]{\textcolor[rgb]{0.00,0.00,0.81}{{#1}}}
\newcommand{\CharTok}[1]{\textcolor[rgb]{0.31,0.60,0.02}{{#1}}}
\newcommand{\StringTok}[1]{\textcolor[rgb]{0.31,0.60,0.02}{{#1}}}
\newcommand{\CommentTok}[1]{\textcolor[rgb]{0.56,0.35,0.01}{\textit{{#1}}}}
\newcommand{\OtherTok}[1]{\textcolor[rgb]{0.56,0.35,0.01}{{#1}}}
\newcommand{\AlertTok}[1]{\textcolor[rgb]{0.94,0.16,0.16}{{#1}}}
\newcommand{\FunctionTok}[1]{\textcolor[rgb]{0.00,0.00,0.00}{{#1}}}
\newcommand{\RegionMarkerTok}[1]{{#1}}
\newcommand{\ErrorTok}[1]{\textbf{{#1}}}
\newcommand{\NormalTok}[1]{{#1}}
\usepackage{graphicx}
\makeatletter
\def\maxwidth{\ifdim\Gin@nat@width>\linewidth\linewidth\else\Gin@nat@width\fi}
\def\maxheight{\ifdim\Gin@nat@height>\textheight\textheight\else\Gin@nat@height\fi}
\makeatother
% Scale images if necessary, so that they will not overflow the page
% margins by default, and it is still possible to overwrite the defaults
% using explicit options in \includegraphics[width, height, ...]{}
\setkeys{Gin}{width=\maxwidth,height=\maxheight,keepaspectratio}
\ifxetex
  \usepackage[setpagesize=false, % page size defined by xetex
              unicode=false, % unicode breaks when used with xetex
              xetex]{hyperref}
\else
  \usepackage[unicode=true]{hyperref}
\fi
\hypersetup{breaklinks=true,
            bookmarks=true,
            pdfauthor={Arindam Basu},
            pdftitle={Decline in HRT Usage and Breast Cancer Incidence in 17 countries},
            colorlinks=true,
            citecolor=blue,
            urlcolor=blue,
            linkcolor=magenta,
            pdfborder={0 0 0}}
\urlstyle{same}  % don't use monospace font for urls
\setlength{\parindent}{0pt}
\setlength{\parskip}{6pt plus 2pt minus 1pt}
\setlength{\emergencystretch}{3em}  % prevent overfull lines
\setcounter{secnumdepth}{0}

%%% Use protect on footnotes to avoid problems with footnotes in titles
\let\rmarkdownfootnote\footnote%
\def\footnote{\protect\rmarkdownfootnote}

%%% Change title format to be more compact
%\usepackage{titling}

% Create subtitle command for use in maketitle
\newcommand{\subtitle}[1]{
  \posttitle{
    \begin{center}\large#1\end{center}
    }
}

%\setlength{\droptitle}{-2em}
  \title{Decline in HRT Usage and Breast Cancer Incidence in 17 countries}
  \pretitle{\vspace{\droptitle}\centering\huge}
  \posttitle{\par}
  \author{Arindam Basu}
  \preauthor{\centering\large\emph}
  \postauthor{\par}
  \date{}
  \predate{}\postdate{}



\begin{document}

\maketitle


The purpose of this document is to describe the association between the
decline in the contains the R codes for the breast cancer study.

\begin{Shaded}
\begin{Highlighting}[]
\NormalTok{## set the working directory}

\NormalTok{Mywd <-}\StringTok{ }\KeywordTok{setwd}\NormalTok{(}\StringTok{"/Users/arindambose/GitHub/brca-study"}\NormalTok{)}

\NormalTok{## Read the data into R}

\NormalTok{mydata <-}\StringTok{ }\KeywordTok{read.csv}\NormalTok{(}\StringTok{"breastca.csv"}\NormalTok{, }\DataTypeTok{header =} \OtherTok{TRUE}\NormalTok{, }\DataTypeTok{sep =} \StringTok{","}\NormalTok{)}
\NormalTok{mydata$hrtuse2 <-}\StringTok{ }\NormalTok{(mydata$declinehrtuse)^}\DecValTok{2}
\end{Highlighting}
\end{Shaded}

We regressed the decline in breast cancer incidence on the percentage
decline in the usage of HRT in the countries studies. The

\begin{Shaded}
\begin{Highlighting}[]
 \KeywordTok{plot}\NormalTok{(mydata$declinehrtuse, mydata$cadecrease,}
     \DataTypeTok{main =} \StringTok{"Decrease in Breast Cancer Incidence with HRT"}\NormalTok{,}
    \DataTypeTok{ylab =} \StringTok{"Decrease in Breast Ca Incidence"}\NormalTok{,}
    \DataTypeTok{xlab =} \StringTok{"decrease in HRT usage"}\NormalTok{)}

\KeywordTok{abline}\NormalTok{(cahrt <-}\StringTok{ }\KeywordTok{lm}\NormalTok{(mydata$cadecrease ~}\StringTok{ }\NormalTok{mydata$declinehrtuse,  }
                  \DataTypeTok{data =} \NormalTok{mydata))}
\end{Highlighting}
\end{Shaded}

\includegraphics{brcadecline1.jpeg}

\begin{Shaded}
\begin{Highlighting}[]
\KeywordTok{summary}\NormalTok{(cahrt)}
\end{Highlighting}
\end{Shaded}

\begin{verbatim}
## 
## Call:
## lm(formula = mydata$cadecrease ~ mydata$declinehrtuse, data = mydata)
## 
## Residuals:
##     Min      1Q  Median      3Q     Max 
## -17.034  -4.220   1.374   5.311  16.614 
## 
## Coefficients:
##                      Estimate Std. Error t value Pr(>|t|)
## (Intercept)          -0.62479    5.53244  -0.113    0.912
## mydata$declinehrtuse  0.08853    0.11251   0.787    0.445
## 
## Residual standard error: 9.132 on 13 degrees of freedom
##   (6 observations deleted due to missingness)
## Multiple R-squared:  0.04546,    Adjusted R-squared:  -0.02796 
## F-statistic: 0.6192 on 1 and 13 DF,  p-value: 0.4455
\end{verbatim}

As can be seen from the results, there is a small and statistically
non-significant increment in the decline rates in breast cancer
incidence (beta = 0.085, p = 0.45). In addition to this, we also get to
see that for five countries, while HRT usage declined, this was not
followed by corresponding decline in Breast cancer incidence in those
countries. For these countries, there were increment in the incidence
rates of breast cancer.

Next, we remove data where the decrease in breast cancer were
\textless{} 0, that is where we noted that breast cancer incidence
increased even while the rates of usage of HRT declined.

\includegraphics{brcadecline2.jpeg}

\begin{verbatim}
## 
## Call:
## lm(formula = mydata2$cadecrease ~ mydata2$declinehrtuse, data = mydata2)
## 
## Residuals:
##    Min     1Q Median     3Q    Max 
## -7.628 -2.590 -1.878  3.416  9.447 
## 
## Coefficients:
##                       Estimate Std. Error t value Pr(>|t|)  
## (Intercept)             1.3222     3.8064   0.347   0.7385  
## mydata2$declinehrtuse   0.1681     0.0762   2.206   0.0632 .
## ---
## Signif. codes:  0 '***' 0.001 '**' 0.01 '*' 0.05 '.' 0.1 ' ' 1
## 
## Residual standard error: 5.337 on 7 degrees of freedom
## Multiple R-squared:   0.41,  Adjusted R-squared:  0.3258 
## F-statistic: 4.865 on 1 and 7 DF,  p-value: 0.06319
\end{verbatim}

\begin{verbatim}
##    Min. 1st Qu.  Median    Mean 3rd Qu.    Max. 
##   2.500   3.700   6.300   8.744  10.000  21.800
\end{verbatim}

Selecting only those countries that showed decline in breast cancer
incidence and HRT usage as well shows a sharper decline rate (beta =
0.17, p = 0.06). This was further tested by examining by creating a
binary variable on the basis of the median of the rate of decline
(median percentage for the rate of decline in breast cancer incidence
was 6.3\%, and this was used to classify the countries with decline as
high and low ). Alternatively, using the tertiles (33 rd percentiles,
the countries with different decline rates were classified)

\begin{verbatim}
##   highredca declinehrtuse
## 1     FALSE        31.266
## 2      TRUE        60.275
\end{verbatim}

As can be seen, for countries that registered more than 6\% decline in
the breast cancer incidence, the decline in HRT usage was 60.2 percent,
nearly twice that of the countries with relatively smaller rate of
decline in breast cancer incidence.

\begin{verbatim}
## 
##  (0,3.7] (3.7,10]  (10,21] 
##        3        4        1
\end{verbatim}

\begin{verbatim}
##  (0,3.7] (3.7,10]  (10,21] 
## 24.44333 47.86750 67.00000
\end{verbatim}

As can be seen, the decline in HRT usage was around 67\% for the country
with over 10\% decline in breast cancer decline, while for countries
where the decline in breast cancer was less than 4\%, there was a
decline in HRT usage of around 24\%; together, these findings suggest
that with progressive decline in HRT usage, there were corresponding
decline in breast cancer incidence as well. This was examined in the
plot below.

\begin{Shaded}
\begin{Highlighting}[]
\KeywordTok{barplot}\NormalTok{(hrtcadeclevel,}
        \DataTypeTok{col =} \StringTok{"black"}\NormalTok{, }\DataTypeTok{main =} \StringTok{"Bar Plot of % HRT Reduction"}\NormalTok{,}
        \DataTypeTok{xlab =} \StringTok{"Extent of Decrease in Breast Cancer"}\NormalTok{,}
        \DataTypeTok{ylab =} \StringTok{"Extent of Decrease in HRT Usage"}\NormalTok{,}
       \DataTypeTok{names.arg =} \NormalTok{caredcatnames,}
        \DataTypeTok{ylim =} \KeywordTok{c}\NormalTok{(}\DecValTok{0}\NormalTok{, }\DecValTok{70}\NormalTok{))}
\end{Highlighting}
\end{Shaded}

\includegraphics{brcadecline3.jpeg}

\end{document}
